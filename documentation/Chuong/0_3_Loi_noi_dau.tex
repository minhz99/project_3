% !TeX root = ../main.tex

\documentclass[../main.tex]{subfiles}
\begin{document}

\begin{center}
    \Large{\textbf{LỜI NÓI ĐẦU}}\\
\end{center}
\vspace{1cm}

Trong thời đại của cuộc cách mạng công nghiệp 4.0, việc ứng dụng công nghệ số vào các hệ thống truyền thống không chỉ là xu hướng tất yếu mà còn là yêu cầu cấp thiết để nâng cao hiệu quả và tính bền vững. Với tư cách là một sinh viên ngành Kỹ thuật Nhiệt tại Đại học Bách khoa Hà Nội, em nhận thức sâu sắc về vai trò của kỹ sư trong việc chuyển đổi số và hiện đại hóa các hệ thống công nghiệp, đặc biệt là trong bối cảnh Việt Nam đang tăng cường phát triển công nghiệp và hướng tới các mục tiêu phát triển bền vững.

Tháp giải nhiệt, với vai trò thiết yếu trong hầu hết các quy trình công nghiệp từ nhà máy điện đến trung tâm dữ liệu, đại diện cho một lĩnh vực ứng dụng điển hình của công nghệ giám sát thông minh. Tuy nhiên, thực tế cho thấy hầu hết các hệ thống này tại Việt Nam vẫn được vận hành theo các phương pháp truyền thống, thiếu khả năng tối ưu hóa và dự báo, dẫn đến lãng phí năng lượng và chi phí vận hành cao.

Xuất phát từ mong muốn đóng góp vào việc giải quyết những thách thức thực tiễn này, đồ án "Thiết kế hệ thống IoT giám sát và tính toán thông số tháp giải nhiệt" được hình thành với tầm nhìn phát triển một giải pháp công nghệ phù hợp với điều kiện kinh tế - kỹ thuật của Việt Nam. Đây không chỉ là cơ hội để ứng dụng kiến thức lý thuyết vào thực tiễn mà còn là bước đệm quan trọng trong hành trình nghiên cứu và phát triển công nghệ của em.

Đồ án tập trung vào phát triển hệ thống giám sát IoT cho tháp giải nhiệt với chi phí tối ưu, đồng thời xây dựng cơ sở dữ liệu thực nghiệm phục vụ nghiên cứu và cải tiến sau này. Giải pháp dựa trên công nghệ mã nguồn mở, hướng đến tính ứng dụng cao cho cộng đồng nghiên cứu và doanh nghiệp trong nước.

Quá trình thực hiện đồ án không chỉ là hành trình khám phá và ứng dụng kiến thức mà còn là trải nghiệm quý báu giúp em hiểu sâu hơn về thực tiễn kỹ thuật và phát triển tư duy sáng tạo trong việc giải quyết các vấn đề công nghệ. Đồng thời, đây cũng là cơ hội để em đóng góp một phần nhỏ vào sự nghiệp phát triển công nghệ của đất nước trong bối cảnh chuyển đổi số quốc gia.

Đồ án được thực hiện dưới sự hướng dẫn khoa học của TS. Đỗ Mạnh Hùng và nhận được sự hỗ trợ về cơ sở vật chất từ Phòng thí nghiệm Nghiên cứu Nhiên liệu và Năng lượng sạch (FCE Lab), Trường Cơ khí, Đại học Bách khoa Hà Nội. Em xin bày tỏ lòng biết ơn sâu sắc đến sự định hướng khoa học chính xác, các điều kiện nghiên cứu thuận lợi và những đóng góp quý báu từ các thầy cô và các bạn sinh viên tại phòng thí nghiệm, những người đã tận tâm hướng dẫn và hỗ trợ em trong suốt quá trình thực hiện đề tài.

Thông qua đồ án này, em hy vọng đóng góp một phần nhỏ vào sự nghiệp ứng dụng công nghệ cao phục vụ phát triển kinh tế - xã hội của đất nước, đồng thời tạo tiền đề cho các nghiên cứu sâu hơn về ứng dụng IoT trong lĩnh vực năng lượng và môi trường. Mặc dù đã nỗ lực hết mình trong quá trình nghiên cứu, do hạn chế về thời gian, kinh nghiệm và điều kiện thực nghiệm, đồ án không tránh khỏi những thiếu sót nhất định. Em mong nhận được những ý kiến đóng góp, phê bình xây dựng từ hội đồng đánh giá và các chuyên gia để hoàn thiện hơn nữa nội dung nghiên cứu và định hướng phát triển trong tương lai.

\end{document}