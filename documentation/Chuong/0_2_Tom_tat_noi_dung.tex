% !TeX root = ../main.tex

\documentclass[../main.tex]{subfiles}
\begin{document}

\begin{center}
    \Large{\textbf{TÓM TẮT NỘI DUNG}}\\
\end{center}
\vspace{1cm}
Trong bối cảnh chuyển đổi số và mục tiêu phát triển bền vững của ngành năng lượng, nhu cầu giám sát–tối ưu hóa hệ thống giải nhiệt bằng công nghệ Internet vạn vật - IoT trở nên cấp thiết. Các nhà máy điện, trung tâm dữ liệu và cơ sở công nghiệp chịu áp lực tiết kiệm năng lượng và giảm phát thải; chi phí làm mát có thể chiếm 30–50\% chi phí vận hành hằng ngày. Dù tháp giải nhiệt là tải quan trọng, nhiều hệ thống vẫn vận hành theo kinh nghiệm và các tham số cố định, thiếu giám sát thông minh, dẫn tới lãng phí năng lượng và tăng chi phí bảo trì. Các giải pháp giám sát truyền thống lại đắt đỏ và đòi hỏi chuyên môn cao, vượt quá khả năng của doanh nghiệp vừa và nhỏ—nhóm chiếm tỷ trọng lớn nhưng hạn chế nguồn lực. Trước thực trạng này, đồ án đề xuất một hệ thống giám sát thông minh cho tháp giải nhiệt dựa trên IoT, với mục tiêu tính toán theo thời gian thực hiệu suất và công suất giải nhiệt. Hệ thống theo dõi liên tục các thông số thiết yếu (nhiệt độ, độ ẩm không khí; nhiệt độ nước vào/ra; lưu lượng tuần hoàn) để suy ra năng lực làm mát và các chỉ số hiệu quả, hướng tới một giải pháp chi phí hợp lý nhờ vi điều khiển và cảm biến chính xác, dễ triển khai trong thực tế. Phạm vi nghiên cứu gồm hai phần: (i) thiết kế và triển khai kiến trúc IoT thời gian thực có khả năng thu thập dữ liệu liên tục, xử lý–tính toán tại chỗ các chỉ số như công suất giải nhiệt, hiệu suất làm mát và KPI vận hành; (ii) chế tạo mô hình tháp giải nhiệt mini để tích hợp, kiểm chứng và quan sát sự suy giảm công suất/hiệu suất theo thời gian, đồng thời khảo sát ảnh hưởng của điều kiện môi trường (nhiệt độ, độ ẩm) nhằm tạo dữ liệu thực nghiệm phục vụ phát triển thuật toán. Đóng góp kỳ vọng gồm: xây dựng nền tảng giám sát IoT tối ưu chi phí dựa trên công nghệ mã nguồn mở (ESP32, InfluxDB, Grafana); phát triển thuật toán tính toán thời gian thực chuyên biệt cho tháp giải nhiệt; thiết kế mô hình tháp mini đa cảm biến có khả năng mô phỏng chân thực; đề xuất phương pháp đánh giá hiệu suất dựa trên dữ liệu thời gian thực; và xây dựng cơ sở dữ liệu thực nghiệm trong khí hậu nhiệt đới—có ý nghĩa cho ứng dụng tại Việt Nam và Đông Nam Á, góp phần giảm tiêu thụ năng lượng và chi phí vận hành.
% \begin{flushright}
% Sinh viên thực hiện\\
% \begin{tabular}{@{}c@{}}
% \textit{(Ký và ghi rõ họ tên)}
% \end{tabular}
% \end{flushright}

\end{document}